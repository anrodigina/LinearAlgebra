\documentclass{article}     
\usepackage[utf8]{inputenc} 
\usepackage{amsfonts}
\usepackage[left=4cm,right=4cm,
    top=3cm,bottom=4cm,bindingoffset=0cm]{geometry}
\sloppy
\usepackage[T2A]{fontenc}
\usepackage{amsmath,amsfonts,amssymb,amsthm,mathtools}
\title{Определения и формулировки теорем к коллоквиуму по линейной алгебре}  
\author{(часть определений - копипаста определений прошлого года!!1!)}     
\date{20 мая 2017}   

\usepackage{graphicx}
\graphicspath{{pictures/}}
\DeclareGraphicsExtensions{.pdf,.png,.jpg}

\newcommand{\ip}[2]{(#1, #2)}
                             
\begin{document}             
\maketitle
\noindent \textbf{1. Сумма двух подпространств векторного пространства.}\\

Пусть $V$ --- конечномерное векторное пространство, а $U$ и $W$ --- его подпространства.

Сумма подпространств $U$ и $W$ --- это множество
\[
U+W = \{u + w\ |\ u \in U, w \in W\},
\]
которое является подпространством векторного пространства $V$.
\newline
\newline
\textbf{2. Теорема о связи размерности суммы двух подпространств с размерностью их пересечения.}\\
Пусть $V$ --- конечномерное векторное пространство, а $U$ и $W$ --- его подпространства.

$\dim \left(U \cap W\right) = \dim U + \dim W - \dim \left(U+W\right)$
\\
\textbf{Прямая сумма двух подпространств векторного пространства.}\\
Пусть $V$ --- конечномерное векторное пространство, а $U$ и $W$ --- подпространства.

Если $U \cap W = \{0\}$, то $U + W$ называется прямой суммой.
\newline
\newline
\textbf{3. Сумма нескольких подпространств векторного пространства.}\\
Пусть $U_1, \ldots, U_k$ --- подпространства векторного пространства $V$. Суммой нескольких пространств называется $U_1 + \ldots + U_k = \{u_1 + \ldots + u_k \; | \; u_i \in U_i \}$.
\newline
\newline
\textbf{4. Линейная независимость нескольких подпространств векторного пространства.}\\
   Пусть $V$ -- векторное пространство, $U_1, U_2, \dotso, U_m$ - подпространства $V$. Будем говорить, что $U_1, U_2, \dotso, U_m$ - линейно независимые подпространства, если
   $$
       x_1 + x_2 + \dotso + x_m = 0 \Rightarrow x_1 = x_2 = \dotso = 0, ~~~~ x_i \in U_i
   $$
\noindent
\textbf{5. Эквивалентные условия, определяющие линейно независимый набор подпространств векторного пространства.}\\
   Пусть $V$ -- векторное пространство, $U_1, U_2, \dotso, U_m$ - подпространства $V$. Подпространства $U_1, U_2, \dotso, U_m$ - линейно независимые если:
   \begin{enumerate}
       \item Сумма $U_1 + \dotso + U_m$ -- прямая
       \item Если $i$ -- базис $U_i$, то $1 \cup \dotso \cup m$ -- базис $U_1 + \dotso + U_m$
       \item $dim(U_1 + \dotso + U_k) = dim U_1 + \dotso + dim U_m$
   \end{enumerate}
\noindent
\textbf{6. Разложение векторного пространства в прямую сумму подпространств.}\\
   Пусть $V$ -- векторное пространство, $U_1, U_2, \dotso, U_m$ - подпространства $V$. Будем говорить, что $V$ разлагается в прямую сумму подпространств, если $V = U_1 \oplus U_2 \oplus \dotso \oplus U_m$. А каждый вектор $v\in V$ разложим следующим образом $v=u_1+u_2+..u_m,  ~~~~u_i\in U_i$, причем такое разложение единственно.
\newline
\newline
\textbf{7. При каких условиях на подпространства $U_1,U_2$ векторного пространства $V$ имеет место разложение $V = U_1 \oplus U_2$?}\\
1) $U_1 \cap U_2 = \{0\}$\\
2) $V = U_1+U_2$\\
3) $dimV= dimU_1+dimU_2$\\
4) каждый элемент пространства V может быть единственным образом представлен в виде: $v = u_1+u_2,~~~~~~u_1\in U_1, u_2 \in U_2$
\newline
\newline
\textbf{8. Описание всех базисов $n$-мерного векторного пространства в терминах одного базиса и матриц координат.}\\
   Пусть $V$ -- векторное пространство, $dim V = n, e_1, \dotso, e_n$ -- базис. То есть $\forall v \in V \exists! v = x_1e_1 + \dotso + x_ne_n$, где $x_1, \dotso, x_n \in F$ -- координаты вектора $v$ в базисе $(e_1,\dotso, e_n)$. Пусть также есть базис $e'_1,\dotso,e'_n$:
   \begin{gather*}
       e'_1 = c_{11}e_1 + c_{21}e_2 + \dotso + c_{n1}e_n \\
       e'_2 = c_{12}e_1 + c_{22}e_2 + \dotso + c_{n2}e_n \\
       \vdots \\
       e'_n = c_{1n}e_1 + c_{2n}e_2 + \dotso + c_{nn}e_n
   \end{gather*}
   Обозначим матрицу $C = (c_{ij})$. Тогда можно переписать $(e'_1,\dotso,e'_n)$ как $(e_1, \dotso, e_n)\cdot C$.\\ $e'_1, \dotso, e'_n$ образуют базис тогда и только тогда, когда $\det C \ne 0$.
\newline
\newline
\textbf{9. Матрица перехода от одного базиса векторного пространства к другому.}\\
Пусть $V$ --- векторное пространство, $\dim V = n$, $e = (e_1, \ldots, e_n)$ и $e' =(e_1', \ldots, e_n')$ --- базисы в $V$.

Матрицей перехода от базисе $e$ к базису $e'$ называется матрица, по столбцам которой стоят координаты базиса $e'$ в базисе $e$.
\begin{gather*}
e'_j = \sum_{i = 1}^{n} c_{ij}e_i, \quad c_{ij} \in F \\
(e'_1, \ldots, e'_n) = (e_1, \ldots, e_n) \cdot C, \quad C = (c_{ij}) \text{--- матрица перехода}
\end{gather*}
\newline
\newline
\textbf{10. Формула преобразования координат вектора при замене базиса векторного пространства.}\\
Пусть $V$ --- векторное пространство. Формула преобразования координат вектора $v \in V$ при переходе от базиса $e$ к $e'$:
\begin{gather*}
\begin{pmatrix}
x_1 \\
\vdots \\
x_n
\end{pmatrix}
= C 
\begin{pmatrix}
x'_1 \\
\vdots \\
x'_n
\end{pmatrix}
\qquad \text{или} \qquad
x_i = \sum_{j = 1}^{n}c_{ij}x'_j,
\end{gather*}
где $(x_1, \ldots, x_n)$ --- координаты вектора $v$ в базисе $e $, $(x_1', \ldots, x_n')$ --- координаты вектора $v$  в базисе $e '$ и $C$ --- матрица перехода от базиса $e $ к базису $e '$.
\newline
\newline
\textbf{11. Линейное отображение векторных пространств, его простейшие свойства.}\\
Пусть $V$ и $W$ --- два векторных пространства над полем $F$.

Отображение $f : V \rightarrow W$ называется линейным, если:
\begin{enumerate}
\item $f(u_1 + u_2) = f(u_1) + f(u_2), \quad \forall u_1, u_2 \in V$;
\item $f(\alpha u) = \alpha f(u), \quad \forall u \in V,\ \forall \alpha \in F$.
\end{enumerate}
Его простейшие свойства:
\begin{enumerate}
     \item $f(\bar{0}_V)=\bar{0}_w$
     \item $f(-v) = -f(v) ~~\forall v \in V$
\end{enumerate}
\textbf{12. Изоморфизм векторных пространств. Изоморфные векторные пространства.}\\
Пусть $V$ и $W$ --- два векторных пространства над полем $F$.

Отображение $\phi: V \rightarrow W$ называется изоморфизмом, если $\phi$ линейно и биективно. Обозначение: $\phi : V \simeq  W$.

Два векторных пространства $V$ и $W$ называются изоморфными, если существует изоморфизм $\phi: V \simeq W$ (и тогда существует изоморфизм $V \backsimeq W$). Обозначение: $V \simeq W$ или $V \cong W$.
\newline
\newline
\textbf{13. Какими свойствами обладает отношение изоморфности на множестве всех векторных пространств?}\\
Изоморфность --- это отношение эквивалентности на множестве векторных пространств над фиксированным полем F. (т.е. оно рефлексивно, симметрично и транзитивно)
\newline
\newline
\textbf{14. Критерий изоморфности двух конечномерных векторных пространств.}\\
Два конечномерных векторных пространства $V$ и $W$ изоморфны тогда и только тогда, когда $dim V = dim W$.
\newline
\newline
\textbf{15. Матрица линейного отображения.}\\
Пусть $V$ и $W$ --- векторные пространства, $e = (e_1, \ldots, e_n)$ --- базис $V$, $f = (f_1, \ldots, f_m)$ --- базис $W$, $\phi: V \rightarrow W$ --- линейное отображение.

Матрицей линейного отображения $\phi$ в базисах $e$ и $f$ (или по отношению к базисам $e$ и~$f$) называется такая матрица, у которой в $j$-ом столбце выписаны координаты вектора $\phi(e_j)$ в базисе $f$.
\[
\phi(e_j) = a_{1j}f_1 + \ldots + a_{mj}f_m = \sum_{i = 1}^{m}a_{ij}f_i, \quad A = (a_{ij}) \in Mat_{m\times n} \text{ --- матрица $\phi$}
\]
\newline
\newline
\textbf{16. Связь между координатами вектора и его образа при линейном отображении.}\\
Пусть $e = (e_1, e_2...e_n)$ - базис в $V$, $f = (f_1, f_2...f_m)$ - базис в $W$. 
$$v = x_1e_1+x_2e_2+..+x_ne_n$$
$$\varphi \in Hom(V,W),~~~A=A(\varphi, e, f)$$
$$\varphi(v)=y_1f_1+y_2f_2+..+y_mf_m$$
Тогда:
$$\begin{pmatrix} y_1\\y_2\\\vdots\\y_m\end{pmatrix}=A\begin{pmatrix}x_1\\x_2\\\vdots\\x_n\end{pmatrix}
$$
\newline
\newline
\textbf{17. Формула изменения матрицы линейного отображения при замене базисов.}\\
Пусть $e = (e_1, e_2...e_n)$ - базис в $V$, $f = (f_1, f_2...f_m)$ - базис в $W$. А $e' = (e_1', e_2'...e_n')$, $f' = (f_1', f_2'...f_m')$ - другие базисы в $V$ и $W$ соответственно.
$$A=A(\varphi,e,f);~~A'=A(\varphi, e', f')$$
Пусть $e'=eC$, $f'=fD$. Тогда:
$$A'=D^{-1}AC$$
\newline
\newline
\textbf{18. Сумма двух линейных отображений и её матрица. Произведение линейного отображения на скаляр и его
матрица.}\\
Пусть $\phi, \psi \in Hom(V, W)$.

Отображение $\phi + \psi \in Hom(V, W)$ --- это $(\phi + \psi)(v):= \phi(v) + \psi(v)$ -- сумма отображений.

Пусть $e = (e_1, \ldots, e_n)$ --- базис $V$, $f = (f_1, \ldots, f_m)$ --- базис $W$, $\phi, \psi \in Hom(V, W)$. При этом $A_{\phi}$ --- матрица линейного отображения $\phi$, $A_{\psi}$ -Пусть $\phi, \psi \in Hom(V, W)$.

Отображение $\alpha \in F, \alpha\phi \in Hom(V, W)$ --- это $(\alpha\phi)(v) := \alpha(\phi(v))$ -- произведение линейного отображения на скаляр.

Пусть $e = (e_1, \ldots, e_n)$ --- базис $V$, $f= (f_1, \ldots, f_m)$ --- базис $W$, $\phi, \psi \in Hom(V, W)$. При этом $A_{\phi}$ --- матрица линейного отображения $\phi$, $A_{\psi}$ --- матрица для $\psi$,  $A_{\alpha\phi}$ --- для $\alpha\phi$.

Тогда $A_{\alpha\phi} = \alpha A_{\phi}$.-- матрица для $\psi$, $A_{\phi+\psi}$ --- для $\phi + \psi$.

Тогда $A_{\phi+\psi} = A_{\phi} + A_{\psi}$.
\newline
\newline
\textbf{19. Композиция двух линейных отображений и её матрица.}\\
Возьмем три векторных пространства --- $U, V$ и $W$ размерности $n, m$ и $k$ соответственно, и их базисы $e, f$ и $g$. Также рассмотрим цепочку линейных отображений $U \xrightarrow{\psi} V \xrightarrow{\phi} W$. 

Отображение $\phi\circ\psi \in Hom(U, W)$ -- это $(\phi\circ\psi)(v) := \phi(\psi(v))$ -- композиция линейных отображений.

Пусть $A$ --- матрица $\phi$ в базисах $f$ и $g$, $B$ --- матрица $\psi$ в базисах $e$ и $f$, $C$ --- матрица $\phi\circ\psi$ в базисах $e$ и $g$.

Тогда $C = AB$.
\newline
\newline
\textbf{20. Ядро и образ линейного отображения.}\\
Пусть $V$ и $W$ --- векторные пространства, $\phi: V \rightarrow W$ --- линейное отображение.

\textit{Ядро $\phi$} --- это множество $Ker\phi := \{v \in V \mid \phi(v) = 0 \}$.

\textit{Образ $\phi$} --- это множество $Im \phi := \{w \in W \mid \exists v \in V : \phi(v) = w \}$.
\newline
\newline
\textbf{21. Критерий инъективности линейного отображения в терминах его ядра. Критерий изоморфности линейного
отображения в терминах его ядра и образа.}\\
Пусть $\phi\colon V \rightarrow W$ --- линейное отображение.

Отображение $\phi$ инъективно тогда и только тогда, когда $Ker \phi = \{0\}$.\\
Отображение $\phi$ изоморфно тогда и только тогда, когда $Ker \phi = \{0\}$ и $Im\phi = W$.
\newline
\newline
\textbf{22. Связь между рангом матрицы линейного отображения и размерностью его образа.}\\
Пусть $V, W$ --- векторные пространства, $e = (e_1, \ldots, e_n)$ --- базис $V$, $f= (f_1, \ldots, f_m)$ --- базис $W$, $A$ --- матрица $\phi$ по отношению к $e, f$.
$$dimIm\phi = rkA$$
\newline
\newline
\textbf{23. Оценки на ранг произведения двух матриц.}\\
Пусть $A \in Mat_{k\times m},\ B\in Mat_{m\times n}$. Тогда $rk AB \leqslant min\{rk A, rk B\}$.
\newline
\newline
\textbf{24. Каким свойством обладает набор векторов, дополняющих базис ядра линейного отображения до базиса всего пространства?}\\
Пусть  $\mathbb{e} = (e_1, \ldots, e_n)$ --- базис $V$, такой что $(e_1, \ldots, e_k)$ --- базис ядра $Ker\varphi$. Тогда $\varphi(e_{k+1}),\ldots\varphi(e_n)$ --- базис $Im\varphi$. 
Замечание: базис в $V$ с указанными условиями всегда существует.
\newline
\newline
\textbf{25. Теорема о связи размерностей ядра и образа линейного отображения.}\\
Пусть $\phi\colon V \rightarrow W$ --- линейное отображение.

Тогда $dim Im \phi = dim V - dim Ker \phi$.
\newline
\newline
\textbf{26. К какому простейшему виду можно привести матрицу линейного отображения путём замены базисов?}\\
$A \in Mat_{m\times n}, ~rkA=r \Rightarrow \exists C\in M_n$, $\exists D \in Mn$ --- обе невырожденные, такие что $D^{-1}AC=B$, где количество единиц на диагонали совпадает с $r$
$$
B=
\begin{pmatrix}
1 & 0 & \cdots & 0 &  \cdots & 0 \\
0 & 1 & \cdots & 0 &  \cdots & 0 \\
\vdots & \vdots & \ddots & \vdots& \cdots & 0\\
0 & 0 & 0 & 1&  \cdots & 0\\
\vdots & \vdots & \vdots & \vdots &  \ddots & 0\\
0 & 0 & 0 & 0 & \cdots &0 
\end{pmatrix}
$$
\newline
\newline
\textbf{27. Линейная функция на векторном пространстве.}\\
Линейной функцией (формой, функционалом) на векторном пространстве $V$ называется всякое линейное отображение $\sigma \colon V \rightarrow F$.
\newline
\newline
\textbf{28. Сопряжённое (двойственное) векторное пространство и его размерность.}\\
Пусть $V$ - векторное пространство. $V^*=Hom(V,F)$ - множество всех линейных функций из  $V$ в $F$ - называется двойственным (или сопряженным) пространством.\\ 
Из общей теории линейных отображений $V^*$ - векторное пространство.
    $$dimV^*=n=dimV \Rightarrow V^* \simeq V$$
\newline
\newline
\textbf{29. Базис сопряжённого пространства, двойственный к данному базису исходного векторного пространства.}\\
Пусть $e = (e_1, \ldots, e_n)$ --- базис $V$. Рассмотрим линейные формы $\epsilon_1, \ldots, \epsilon_n$ такие, что $\epsilon_i(e_j) =~\delta_{ij}$, где $\delta_{ij} =
\begin{cases}
1, & i = j \\
0, & i \neq j
\end{cases}
$ --- символ Кронекера. \\То есть $\epsilon_i = (\delta_{i1}, \ldots, \delta_{ii}, \ldots, \delta_{in}) = (0, \ldots, 1, \ldots, 0)$.

Тогда $(\epsilon_1, \ldots, \epsilon_n)$ --- базис в $V^*$, называющийся двойственным (сопряжённым) к базису $e$.
\newline
\newline
\textbf{30. Билинейная форма на векторном пространстве.}\\
Билинейной функцией (формой) на векторном пространстве $V$ называется всякое билинейное отображение $\beta \colon V \times V \rightarrow F$. То есть это отображение, линейное по каждому аргументу:
\begin{enumerate}
\item $\beta(x_1 + x_2, y) = \beta(x_1, y) + \beta(x_2, y)$; 
\item $\beta(\lambda x, y) = \lambda\beta(x, y)$;
\item $\beta(x, y_1 + y_2) = \beta(x, y_1) + \beta(x, y_2)$;
\item $\beta(x, \lambda y) = \lambda\beta(x, y)$.
\end{enumerate}
\textbf{31. Матрица билинейной формы.}\\
Пусть $V$ --- векторное пространство, $\dim V < \infty$, $\beta \colon V \times V \rightarrow F$ --- билинейная функция.

Матрицей билинейной функции в базисе $e = (e_1, \ldots, e_n)$ называется матрица $B = (b_{ij})$, где $b_{ij}= \beta(e_i, e_j)$.
\newline
\newline
\textbf{32. Формула для вычисления значений билинейной формы в координатах.}\\
Пусть $(e_1, \ldots, e_n)$ --- базис $V$, $\beta \colon V \times V \to F$ --- билинейная функция, $B$ --- ее матрица в базисе $e $. Тогда для некоторых векторов $x = x_1e_1 + \ldots + x_ne_n \in V$ и $y = y_1e_1 + \ldots + y_ne_n \in V$ верно, что:
$$
\beta(x, y) = (x_1, \ldots, x_n)B \begin{pmatrix}y_1\\ \vdots \\ y_n \end{pmatrix}
$$
\textbf{33. Формула изменения матрицы билинейной формы при переходе к другому базису.}\\
Пусть $e = (e_1, \ldots, e_n)$ и $e' = (e'_1, \ldots, e'_n)$ --- два базиса $V$, $\beta$ --- билинейная функция на $V$. Пусть также $e' = e C$, где $C$ --- матрица перехода, также $B(\beta, e) = B$ и $B(\beta, e') = B'$.

Тогда $B' = C^TBC$.
\newline
\newline
\textbf{34. Ранг билинейной формы.}\\
Пусть $B(\beta, e)$ -- матрица билинейной функции $\beta$ в базисе $e$.

Число $rk B$ называется рангом билинейной функции $\beta$. Обозначение: $rk \beta$.
\newline
\newline
\textbf{35. Симметричная билинейная форма.}\\
Билинейная функция $\beta$ называется симметричной, если $\beta(x, y) =\beta(y, x)$ для любых $x, y \in V$.
\newline
\newline
\textbf{36. Квадратичная форма.}\\
Пусть $\beta \colon V\times V \rightarrow F$ --- билинейная функция. Тогда функция $Q_\beta \colon V \rightarrow F$, заданная формулой $Q_\beta(x) = \beta(x, x)$, называется квадратичной формой (функцией), ассоциированной с билинейной функцией $\beta$.
\newline
\newline
\textbf{37. Соответствие между симметричными билинейными формами и квадратичными формами.}\\
Пусть $\beta \colon V\times V \rightarrow F$ --- симметричная билинейная функция, где $F$ --- поле, в котором $0 \neq 2$ (то есть можно делить на два).

Отображение $\beta(x, y) \mapsto Q_\beta(x) = \beta(x, x)$ является биекцией между симметричными билинейными функциями на $V$ и квадратичными функциями на $V$.
\newline
\newline
\textbf{38. Симметризация билинейной формы.}\\
Симметричная билинейная функция $\delta(x, y) = \frac{1}{2}\left( \beta(x,y)+\beta(y,x)  \right)$ называется симметризацией квадратичной формы $Q$.
\newline
\newline
\textbf{39. Поляризация квадратичной формы.}\\
Симметричная билинейная функция $\beta(x, y) = \frac{1}{2}\left( Q(x + y) - Q(x) - Q(y)  \right)$ называется поляризацией квадратичной формы $Q$.
\newline
\newline
\textbf{40. Матрица квадратичной формы.}\\
Пусть $V$ --- векторное пространство, $\dim V < \infty$.

Матрицей квадратичной формы $Q \colon V \to F$   в базисе $e$ называется матрица  соответствующей ей симметричной билинейной функции $\beta \colon V \times V \rightarrow F$ в том же базисе. 
\newline
\newline
\textbf{41. Канонический вид квадратичной формы.}\\
Квадратичная форма $Q$ имеет в базисе $e = (e_1, \ldots, e_n)$ канонический вид, если для любого вектора $x = x_1e_1 + \ldots + x_ne_n$ верно, что $Q(x) = a_1x_1^2 +\ldots + a_nx_n^2$, где $a_i \in F$. Иными словами, она имеет диагональную матрицу.
\newline
\newline
\textbf{42. Нормальный вид квадратичной формы над R.}\\
Квадратичная форма $Q$ имеет нормальный вид в базисе $e = (e_1,..,e_n)$, если для любого вектора $x = x_1e_1 + \ldots + x_ne_n$ верно, что $Q(x) = a_1x_1^2 +\ldots + a_nx_n^2$, причем $a_i \in \{-1, 0, 1\}$.  
\newline
\newline
\textbf{43. Индексы инерции квадратичной формы над R.}\\
Пусть $Q$ --- квадратичная форма над $\mathbb{R}$, которая в базисе $e =  (e_1,..,e_n)$ имеет нормальный вид: 
$$
Q(x_1, \ldots, x_n) = x_1^2 + \ldots + x_s^2 - x_{s + 1}^2 - \ldots - x_{s + t}^2,
$$
где $s$ --- это количество положительных слагаемых, а $t$ --- отрицательных. Тогда:

\begin{enumerate}
\item $i_+ := s$ --- положительный индекс инерции;
\item $i_- := t$ --- отрицательный индекс инерции;
\item $i_0 := n - s - t$ --- нулевой индекс инерции.
\end{enumerate}
\textbf{44. Закон инерции для квадратичной формы над R.}\\
Индексы инерции не зависят от выбора базиса, в котором квадратичная форма $Q$ имеет нормальный вид.
\newline
\newline
\textbf{45. Положительно/неотрицательно определённая квадратичная форма.}\\
Квадратичная форма $Q$ называется:
\begin{enumerate}
\item положительно определенной, если $Q(x) > 0\ \forall x \neq 0$. Обозначение: $Q > 0$;
\item неотрицательно определенной, если $Q(x) \geqslant 0\ \forall x$. Обозначение: $Q \geqslant 0$.
\end{enumerate}
\textbf{46. Отрицательно/неположительно определённая квадратичная форма.}\\
Квадратичная форма $Q$ называется:
\begin{enumerate}
\item отрицательно определенной, если $Q(x) < 0\  \forall x \neq 0$. Обозначение: $Q < 0$;
\item неположительно определенной, если $Q(x) \leqslant 0\ \forall x$. Обозначение: $Q \leqslant 0$.
\end{enumerate}
\textbf{47. Неопределённая квадратичная форма.}\\
Квадратичная форма $Q$ называется неопределенной, если существуют такие векторы $x, y$, что $Q(x) > 0$ и $Q(y) < 0$.
\newline
\newline
\textbf{48. Следствие метода Якоби о нахождении индексов инерции квадратичной формы.}\\
Пусть $\delta_k\neq0 ~~~\forall k=1..n$. Тогда $i_-$ равен числу перемен знака в последовательности $1,~\delta_1,~\delta_2..\delta_n$ 
\newline
\newline
\textbf{49. Критерий Сильвестра положительной определённости квадратичной формы.}\\
Квадратичная форма $Q$ является положительно определенной тогда и только тогда, когда $\delta_i > 0$  для всех $i$.

Здесь $\delta_i$ --- угловой минор $i\times i$-подматрицы матрицы соответствующей симметричной билинейной функции $\beta$ в некотором базисе; $\delta_0 = 1$.
\newline
\newline
\textbf{50. Критерий отрицательной определённости квадратичной формы.}\\
Квадратичная форма $Q$ является отрицательно определенной тогда и только тогда,\\ когда:
$$
\begin{cases}
	\delta_i < 0, & 2 \nmid i \\
	\delta_i > 0, & 2 \mid i
\end{cases}.
$$
Здесь $\delta_i$ --- угловой минор $i\times i$-подматрицы матрицы соответствующей симметричной билинейной функции $\beta$ в некотором базисе; $\delta_0 = 1$.
\newline
\newline
\textbf{51. Евклидово пространство.}\\
Евклидово пространство --- это векторное пространство $\mathbb{E}$ над полем $\mathbb{R}$, на котором задана положительно определённая симметрическая билинейная функция $(\cdot, \cdot)$, которую мы будем называть скалярным произведением.
\newline
\newline
\textbf{52. Длина вектора в евклидовом пространстве.}\\
Пусть $\mathbb{E}$ --- евклидово пространство, $x\in \mathbb{E}$. Тогда длиной вектора называют величину $|x| = \sqrt{(x,x)}$.
\newline
\newline
\textbf{53. Неравенство Коши-Буняковского.}\\
Пусть $\mathbb{E}$ --- евклидово пространство, $x, y \in \mathbb{E}$. Тогда $|(x,y)| \leqslant |x||y|$, причём знак равенства возможен тогда и только тогда, когда $x$ и $y$ пропорциональны.
\newline
\newline
\textbf{55. Матрица Грама системы векторов евклидова пространства.}\\
Пусть $\mathbb{E}$ --- евклидово пространство.
Матрица Грама системы $v_1, \ldots, v_k \in \mathbb{E}$ это
	$$G(v_1,\ldots, v_k) := (g_{ij}),\quad g_{ij} = (v_i,v_j).$$
\newline
\newline
\textbf{56. Свойства определителя матрицы Грама.}\\
Пусть $\mathbb{E}$ --- евклидово пространство, $G(v_1,\ldots, v_k)$ --- матрица Грама. Тогда:
\begin{enumerate}
	\item $\det G(v_1, \ldots, v_k) \geqslant 0$;
	\item $\det G(v_1, \ldots, v_k) = 0$ тогда и только тогда, когда $v_1, \ldots, v_k$ линейно зависимы.
\end{enumerate}
\textbf{57. Ортогональное дополнение подмножества евклидова пространства.}\\
Пусть $S$ --- произвольное подпространство евклидова пространства $\mathbb{E}$. Ортогональным дополнением к $S$ называется множество $S^{\perp} = \{x\in \mathbb{E}\; |\; (x,y) = 0\;\forall y \in S\}$.
\newline
\newline
\textbf{58. Чему равна размерность ортогонального дополнения к подпространству?}\\
Пусть $S$ --- произвольное подпространство евклидова пространства $\mathbb{E}$. Тогда
$$dimS^{\perp}=n-dimS$$
где n - размерность $\mathbb{E}$
\newline
\newline
\textbf{59. Каким свойством обладают подпространство евклидова пространства и его ортогональное дополнение?}\\
Для каждого подпространства $S \subseteq\mathbb{E}$ выполняется следующее:
$$\mathbb{E}=S\oplus S^{perp}$$
Это означает, что $\forall v \in \mathbb{E}~~~\exists!x\in S, y \in S^{\perp}$, такой что $v = x+y$
\newline
\newline
\textbf{60. Ортогональная проекция вектора на подпространство.}
Пусть $S$ --- подпространство евклидова пространства $\mathbb{E}$. Тогда любой вектор $x \in E$ единственным образом разбивается на сумму $x = y + z$, где $y \in S$ и $z \in S^\perp$. 

Вектор $y$ называется ортогональной проекцией вектора $x$ на подпространство $S$. Обозначение: $pr_S x$. 
\newline
\newline
\textbf{61. Ортогональная составляющая вектора относительно подпространства.}\\
Пусть $S$ --- подпространство евклидова пространства $\mathbb{E}$. Тогда любой вектор $x \in E$ единственным образом разбивается на сумму $x = y + z$, где $y \in S$ и $z \in S^\perp$. 

Вектор $z$ называется ортогональной составляющей вектора $x$ относительно (вдоль) подпространства $S$. Обозначение: $ort_S x$.
\newline
\newline
\textbf{62. Формула для ортогональной проекции вектора на подпространство в $\mathbb{R}^n$, заданное своим базисом.}\\
Пусть $\mathbb{E}=\mathbb{R}^n$ со стандартным скалярным произведением. $S\subseteq \mathbb{R}^n$ - подпространство. Базис $a_1..a_k$ в $S$ образует матрицу $A\in Mat_{n \times k}(\mathbb{R})$, где $A^{(i)}=a_i$. Тогда
$$\forall v \in \mathbb{E}~~~~~pr_sv=A(A^TA)^{-1}A^Tv$$
\newline
\newline
\textbf{63. Ортогональная система векторов. Ортогональный базис.}\\
Система ненулевых векторов $v_1..v_k \in \mathbb{E}$ называется ортогональной, если $(v_i,v_j)=0$ при $i\neq j$\\ 
Базис $(e_1, \ldots, e_n)$ в евклидовом пространстве $\mathbb{E}$ называется ортогональным, если \\$(e_i, e_j)= 0\; \forall i\neq j$. Это равносильно тому, что $G(e_1, \ldots, e_n)$ диагональна.
\newline
\newline
\textbf{64. Ортонормированная система векторов. Ортонормированный базис.}\\
Система ненулевых векторов $v_1..v_k \in \mathbb{E}$ называется ортонормированной, если $(v_i,v_j)=0$ при $i\neq j$ и $(v_i,v_i)=|v_i|^2=1~~\forall i$\\
Базис $(e_1, \ldots, e_n)$ в евклидовом пространстве $\mathbb{E}$ называется ортонормированным, если он является ортогональным базисом и дополнительно $(e_i, e_i) = 1\; \forall i$. Это равносильно тому, что $G(e_1, \ldots, e_n) = E$.
\newline
\newline
\textbf{65. Описание всех ортонормированных базисов евклидова пространства в терминах одного такого базиса и матриц перехода.}\\
Пусть $e=(e_1,e_2...e_n)$ - ортонормированный базис. $e'=(e_1',e_2'...e_n')=(e_1,e_2...e_n)C_{e\rightarrow e'}$. Тогда\\
$e'$ --- ортонормированный базис $\Leftrightarrow$ $C^TC=E$. Матрица $C$ называется ортогональной
\newline
\newline
\textbf{66. Ортогональная матрица.}\\
Пусть $e = (e_1, \ldots, e_n)$ и $e' = (e_1', \ldots, e_n')$ --- два ортонормированных базиса в евклидовом пространстве $\mathbb{E}$, причем $(e_1', \ldots, e_n') = (e_1, \ldots, e_n)C$. 

Тогда матрица $C$ называется ортогональной.
\newline
\newline
\textbf{67. Формула для ортогональной проекции вектора на подпространство в терминах его ортогонального базиса.}\\
Пусть $S$ --- подпространство евклидова пространства $\mathbb{E}$, $(e_1, \ldots, e_k)$ --- его ортогональный базис, $x \in \mathbb{E}$.
Тогда $pr_S x = \sum\limits_{i = 1}^{k}\cfrac{(x,e_i)}{(e_i, e_i)}e_i$. В частности, если базис ортонормированный, $pr_S x = \sum\limits_{i = 1}^{k}(x,e_i)e_i$
\newline
\newline
\textbf{68. Теорема Пифагора в евклидовом пространстве.}\\
Если векторы евклидова пространства $x$ и $y$ перпендикулярны, то $|x+y|^2 = |x|^2 + |y|^2$.
\newline
\newline
\textbf{69. Расстояние между векторами евклидова пространства.}\\
Расстоянием между векторами евклидова пространства $x$ и $y$ называется число \\$\rho(x,y) := |x-y|$.
\newline
\newline
\textbf{70. Неравенство треугольника в евклидовом пространстве.}\\
$\forall a,b,c \in \mathbb{E}$:
$$\rho(a,b)+\rho(b,c) \ge \rho(a,c)$$
\newline
\newline
\textbf{71. Теорема о расстоянии между вектором и подпространством в терминах ортогональной составляющей.}
Пусть $U$ --- подпространство евклидова пространства $\mathbb{E}$.

Модуль ортогональной составляющей вектора $x \in E$ относительно подпространства $U$ равен расстоянию от вектора $x$ до подпространства $U$.
$$
\rho (x,U) = |ort_U x|.
$$
\newline
\newline
\textbf{72. Псевдорешение несовместной системы линейных уравнений.}\\
Пусть $A\in Mat_{m\times n}$, $x\in \mathbb{R}^n$ - неизвестный вектор, $b \in \mathbb{R}^m$. У нас есть система линейных уравнений (*): $Ax=B$\\ 
$x_0 \in \mathbb{R}^n$ - решение - СЛУ $\Leftrightarrow$ $Ax_0=b$ $\Leftrightarrow$ $Ax_0-b=0$  $\Leftrightarrow$ $|Ax_0-b|=0$.\\ 
В случае, когда СЛУ (*) несовместна, набор $x_0$ называют псевдорешением, если $\rho(Ax_o,b)$ минимально среди всех $x \in \mathbb{R}^n$, то есть $\rho(Ax_0, b)=min(Ax,b)$
\newline
\newline
\textbf{73. Формула для расстояния от вектора до подпространства в терминах матриц Грама.}\\
Пусть $U$ --- подпространство евклидова пространства $\mathbb{E}$, $x \in \mathbb{E}$, $(e_1, \ldots, e_k)$ --- базис $U$. Тогда 
$(\rho(x,U))^2 = \cfrac{\det G(e_1, \ldots, e_k, x)}{\det G(e_1, \ldots, e_k)}$.
1) $x_0$ - псевдорешение для (*) $\Leftrightarrow$ $x_0$ - решение для СЛУ $Ax=pr_sb$
2) Если столбцы матрицы $A$ линейно независимые, то псевдорешение $x_0$ единственно и может быть найдено по формуле: $x_0=(A^TA)^{-1}Ab$
\newline
\newline
\textbf{74. n-мерный параллелепипед и его объём.}\\
$N$-мерным параллелепипедом, натянутым на векторы $a_1, \ldots, a_n$ евклидова пространства $\mathbb{E}$ называется множество 
$$
P(a_1, \ldots, a_n) := \left\{ x = \sum_{i = 1}^n x_ia_i \mid 0 \leqslant x_i \leqslant 1 \right\}.
$$  
Объем $n$-мерного параллелепипеда $P(a_1, \ldots, a_{n})$ --- это число $vol P(a_1, \ldots, a_{n})$, определяемое рекурсивно следующим образом:
\begin{align*}
n = 1 \quad& vol P(a_1) = |a_1| \\
n > 1 \quad& vol P(a_1, \ldots, a_n) = vol P(a_1, \ldots, a_{n-1})\cdot |h|
\end{align*}
Где $h = ort_{\langle a_1, \ldots, a_{n-1}\rangle}a_n$ --- высота $P(a_1, \ldots, a_n)$.
Пусть $\mathbb{E}$ --- векторное пространство, $(e_1,..,e_n)$ --- его ортогональный базис и $(a_1,..,a_n) = (e_1,..,e_n)A$ для некоторой матрицы $A \in M_n(\mathbb{R})$. Тогда $vol P(a_1,..,a_n) = |det A|$.
\newline
\newline
\textbf{75. В каком случае два базиса евклидова пространства называются одинаково ориентированными?}\\
$e=(e_1,e_2...e_n),~~e'=(e_1',e_2'...e_n)$ - два базиса в $\mathbb{E}$. При этом $C$ - матрица перехода: $e'=eC$\\
Базисы $e$ и $e'$ называются одинаково ориентированными, если $detC>0$
\newline
\newline
\textbf{76. Смешанное произведение векторов трёхмерного евклидова пространства, формула для его вычисления в терминах координат в правом ортонормированном базисе.}\\
Пусть $a,b,c \in \mathbb{E}$. Смешанным произведением векторов называется величина $(a,b,c)=Vol(a,b,c)$. Если $(e_1,e_2,e_3)$ - правый ортогональный базис:
$$a = a_1e_1+a_2e_2+a_3e_3$$
$$b = b_1e_1+b_2e_2+b_3e_3$$
$$c = c_1e_1+c_2e_2+c_3e_3$$
тогда $$(a,b,c)=
\begin{vmatrix}
a_1 & a_2 & a_3\\
b_1 & b_2 & b_3\\
c_1 & c_2 & c_3
\end{vmatrix}$$
\newline
\newline
\textbf{77. Критерий компланарности трёх векторов трёхмерного евклидова пространства.}\\
$a,b,c \in \mathbb{E}$ - компланарны (линейно независимы) $\Leftrightarrow$ (a,b,c)=0
\newline
\newline
\textbf{78. Векторное произведение в трёхмерном евклидовом пространстве.}\\
   Векторным произведением вектора $a$ на вектор $b$ в пространстве $\mathbb{R}^3$ называется вектор $c = a \times b = [a, b]$, удовлетворяющий следующим требованиям:
   \begin{enumerate}
       \item $|c| = |a||b|\sin \alpha$, $\alpha$ - угол между $a$ и $b$
       \item вектор $c$ ортогонален каждому из векторов $a$ и $b$
       \item вектор $c$ направлен так, что тройка векторов $a, b, c$ имела одинаковую ориентацию с пространством
   \end{enumerate}
\noindent
\textbf{79. Критерий коллинеарности двух векторов трёхмерного евклидова пространства.}\\
   Два вектора $a$ и $b$ коллинеарны тогда и только тогда, когда $[a, b] = 0$.
\newline
\newline
\textbf{80. Выражение смешанного произведения через векторное и скалярное в трёхмерном евклидовом пространстве.}\\
   Смешанное произведение векторов $a, b, c$:
   $$
       (a, b, c) = (a, [b, c])
   $$
   Геометрический смысл: Модуль смешанного произведения численно равен объёму параллелепипеда, образованного векторами $a, b, c$.
\newline
\newline
\textbf{81. Формула для двойного векторного произведения в трёхмерном евклидовом пространстве.}\\
   Двойное векторное произведение векторов $a, b, c$:
   $$
       [a, b, c] = [a, [b, c]]
   $$
   Для двойного векторного произведения справедлива формула
   $$
       [a, b, c] = [a, [b, c]] = (a, c)b - (a, b)c
   $$
\newline
\textbf{82. Формула для вычисления векторного произведения в терминах координат в правом ортонормированном базисе.}\\
   Пусть $(e_1, e_2, e_3)$ -- правый ортонормированный базис. Тогда
   $$
       x \times y = (x_2y_3 - x_3y_2)e_1 - (x_1y_3 - x_3y_1)e_2 + (x_1y_2 - x_2y_1)e_3
   $$
   Легко видеть, что правая часть -- результат разложения по первой строке "определителя"
   $$
       \begin{vmatrix}
           e_1 & e_2 & e_3 \\
           x_1 & x_2 & x_3 \\
           y_1 & y_2 & y_3
       \end{vmatrix}
   $$
\newline
\textbf{83. Линейное многообразие. Характеризация линейных многообразий как сдвигов подпространств.}\\
   Линейным многообразием в $\mathbb{R}^n$ называется множество всех решений какой-то СЛУ. \\
   Множество $S \in \mathbb{R}^n$ -- линейное многообразие тогда и только тогда, когда $S = v_0 + V$, где $v_0 \in \mathbb{R}^n$, а $V$ -- подпространство в $\mathbb{R}^n$. Иными словами, линейное многообразие есть сдвиг некоего подпространства, и наоборот. Вектор $v_0$ называется вектором сдвига, а $V$ -- направляющим подпространством.
\newline
\newline
\textbf{84. Критерий равенства двух линейных многообразий. Направляющее подпространство и размерность линейного
многообразия.}\\
   Линейные многообразия $S_1 = v_1 + L_1$ и $S_2 = v_2 + L_2$ равны тогда и только тогда, когда $L_1 = L_2$ и $v_2 - v_1 \in L_2$. \\
   Размерность линейного многообразия $S = v_0 + V$ из $\mathbb{R}^n$ -- число $\dim S = \dim V$. Иными словами, размерность линейного многообразия суть размерность направляющего подпространства.
\newline
\newline
\textbf{85. Теорема о плоскости, проходящей через $k + 1$ точку в $\mathbb{R}^n$.}\\
   \begin{enumerate}
       \item Через любые $k + 1$ точек в $\mathbb{R}^n$ проходит плоскость размерности $\le k$. \\
       \item Если $k + 1$ точек не лежат в плоскости размерности $< k$, то через них проходит ровно одна плоскость размерности $k$.
   \end{enumerate}
\noindent
\textbf{86. Три способа задания прямой в $\mathbb{R}^2$. Уравнение прямой в $\mathbb{R}^2$, проходящей через две различные точки.}\\
   Способы задания прямой на плоскости:
   \begin{enumerate}
       \item Общее уравнение прямой \\
           $$
               Ax + By + C = 0
           $$
       \item Каноническое уравнение прямой, проходящей через точку $(x_0, y_0)$ параллельно вектору $\{l, m\}$ \\
           $$
               \frac{x - x_0}{l} = \frac{y - y_0}{m}
           $$
       \item Параметрическое уравнение прямой, проходящей через точку $(x_0, y_0)$ параллельно вектору $\{l, m\}$ \\
           $$
               \begin{cases}
                   x = x_0 + lt \\
                   y = y_0 + mt
               \end{cases}
           $$
   \end{enumerate}
   Уравнение прямой, проходящей через две точки:
   $$
       \frac{y - y_1}{y_2 - y_1} = \frac{x - x_1}{x_2 - x_1}
   $$
\newline
\newline
\textbf{87. Три способа задания плоскости в $\mathbb{R}^3$. Уравнение плоскости в $\mathbb{R}^3$, проходящей через три точки, не лежащие на одной прямой.}
   \begin{enumerate}
       \item Общее уравнение плоскости \\
               $$
                   Ax + By + Cz + D = 0
               $$
       \item Уравнение плоскости в отрезках\\
       $$
           \frac{x}{a} + \frac{y}{b} + \frac{z}{c} = 1
       $$
       \item Нормальное уравнение плоскости\\
       $$
           x\cos\alpha + y\cos\beta + z\cos\gamma - p = 0
       $$
       Где $\cos\alpha, \cos\beta, \cos\gamma$ -- направляющие косинусы нормального вектора данной плоскости единичной длины, а $p$ -- неотрицательное число, равное расстоянию от начала координат до плоскости.
   \end{enumerate}
   Уравнение плоскости в $\mathbb{R}^3$, проходящей через три точки, не лежащие на одной прямой:
   $$
       \begin{vmatrix}
           x - x_1 & y - y_1 & z - z_1 \\
           x_2 - x_1 & y_2 - y_1 & z_2 - z_1 \\
           x_3 - x_1 & y_3 - y_1 & z_3 - z_1
       \end{vmatrix} = 0
   $$
\newline
\newline
\textbf{88. Три способа задания прямой в $\mathbb{R}^3$. Уравнения прямой в $\mathbb{R}^3$, проходящей через две различные точки.}
   \begin{enumerate}
       \item Общее уравнение прямой \\
           $$
               \begin{cases}
                   A_1x + B_1y + C_1z + D_1 = 0 \\
                   A_2x + B_2y + C_2z + D_2 = 0
               \end{cases}
           $$
       \item Каноническое уравнение прямой, проходящей через точку $(x_0, y_0, z_0)$ параллельно вектору $\{l, m, n\}$ \\
           $$
               \frac{x - x_0}{l} = \frac{y - y_0}{m} = \frac{z - z_0}{n}
           $$
       \item Параметрическое уравнение прямой, проходящей через точку $(x_0, y_0)$ параллельно вектору $\{l, m\}$ \\
           $$
               \begin{cases}
                   x = x_0 + lt \\
                   y = y_0 + mt \\
                   z = z_0 + nt
               \end{cases}
           $$
   \end{enumerate}
   Уравнения прямой в $\mathbb{R}^3$, проходящей через две различные точки:
   $$
       \frac{x - x_1}{x_2 - x_1} = \frac{y - y_1}{y_2 - y_1} = \frac{z - z_1}{z_2 - z_1}
   $$
\newline
\newline
\textbf{89. Случаи взаимного расположения двух прямых в $\mathbb{R}^3$.} \\
   Две прямые $a$ и $b$ пространства могут:
   \begin{enumerate}
       \item совпадать: $a = b$
       \item быть параллельными: $a || b$
       \item пересекаться в точке $M = a \cap b$
       \item скрещиваться
   \end{enumerate}
\noindent
\textbf{90. Случаи взаимного расположения трёх попарно различных плоскостей в $\mathbb{R}^3$.}
   \begin{enumerate}
       \item Плоскости имеют одну общую точку
       \item Плоскости попарно пересекаются, но не имеют общей точки
       \item Плоскости проходят через одну прямую
       \item Две плоскости параллельны и третья их пересекает
       \item Три плоскости попарно параллельны
   \end{enumerate}
\noindent
\textbf{91. Формула для расстояния от точки до прямой в $\mathbb{R}^3$.}\\
   Пусть дана прямая, проходящяя через точку $M_1(x_1, y_1, z_1)$ параллельно вектору $a = \{l, m, n\}$ и точка $M_0(x_0, y_0, z_0)$. Расстояние от $M_0$ до прямой вычисляется по формуле:
   $$
       d = \frac{|(M_1 - M_0) \times a|}{|a|}
   $$
\newline
\newline
\textbf{92. Формула для расстояния от точки до плоскости в $\mathbb{R}^3$.}\\
   Пусть дана плоскость $Ax + By + Cz + D = 0$ и точка $M(x_0, y_0, z_0)$. Расстояние от этой точки до плоскости вычисляется по формуле:
   $$
       d = \frac{|Ax_0 + By_0 + Cz_0 + D|}{\sqrt{A^2 + B^2 + C^2}}
   $$
\newline
\newline
\textbf{93. Формула для расстояния между двумя скрещивающимися прямыми в $\mathbb{R}^3$.}\\
   Пусть даны две скрещивающиеся прямые: одна проходит через точку $M_1$ параллельно вектору $a$, а вторая проходит через точку $M_2$ параллельно вектору $b$. Пусть $m = M_2 - M_1$. Тогда расстояния между этими прямыми вычисляется по формуле:
   $$
       d = \frac{|(m, a, b)|}{|a \times b|}
   $$
\newline
\newline
\textbf{94. Линейный оператор.}\\
Пусть $V$ --- конечномерное векторное пространство.

Линейным оператором (или линейным преобразованием) называется всякое линейное отображение $\phi \colon V \rightarrow V$, то есть из $V$ в себя.
\newline
\newline
\textbf{95. Матрица линейного оператора.}
Пусть $V$ --- векторное пространство, $\mathbb{e} = (e_1, \ldots, e_n)$ --- его базис и $\phi$ --- его линейный оператор.

Матрицей линейного оператора $\phi$ называется такая матрица, в $j$-ом столбце которой стоят координаты вектора $\phi(e_j)$ в базисе $e$.
$$
\left(\phi(e_1), \ldots, \phi(e_n)\right) = \left(e_1, \ldots, e_n\right)A, \quad A \text{ --- матрица $\phi$}
$$
\textbf{96. Формула преобразования координат вектора при действии линейного оператора.} \\
   Пусть $\varphi$ - линейный оператор в пространстве $V$, а $A$ - матрица этого оператора в базисе $V$. Тогда
   $$
       \varphi(v) =
       \begin{pmatrix}
           y_1 \\ \vdots \\ y_n
       \end{pmatrix} =
       A \cdot \begin{pmatrix}
           x_1 \\ \vdots \\ x_n
       \end{pmatrix}
   $$
\newline
\newline
\textbf{97. Формула изменения матрицы линейного оператора при переходе к другому базису.}\\
Пусть $\phi$ --- линейный оператор векторного пространства $V$, $A$ --- матрица $\phi$ в базисе $\mathbb{e} = (e_1, \ldots, e_n)$. Пусть $e' = (e_1', \ldots, e_n')$ --- другой базис, причём $(e_1', \ldots, e_n') = (e_1, \ldots, e_n)C$, где $C$ --- матрица перехода, и $A'$ --- матрица $\phi$ в базисе $e'$.

Тогда $A' = C^{-1}AC$.
\newline
\newline
\textbf{98. Подобные матрицы.}\\
Две матрицы $A', A \in M_n(F)$ называются подобными, если существует такая матрица $C \in M_n(F), det C \neq 0$, что $A' = C^{-1}AC$.
\newline
\newline
\textbf{99. Подпространство, инвариантное относительно линейного оператора.}\\
Пусть $\phi\colon V \rightarrow V$ --- линейный оператор.

Подпространство $U \subseteq V$ называется инвариантным относительно $\phi$ (или $\phi$-инвариантным), если $\phi(U)\subseteq U$. То есть $\forall u\in U \colon \phi(u)\in U$. 
\newline
\newline
\textbf{100. Матрица линейного оператора в базисе, дополняющем базис инвариантного подпространства.}\\
Пусть $\phi\colon V \rightarrow V$ --- линейный оператор.

Пусть $U\subset V$ --- $\phi$-инвариантное подпространство. Также пусть $(e_1, \ldots, e_k)$ --- базис в $U$. Дополним его до базиса $V\colon$ $\mathbb{e} = (e_1, \ldots, e_n)$. Тогда
\begin{gather}
    \underbrace{A(\phi,\;\mathbb{e})}_{\text{Матрица с углом нулей}} = \begin{pmatrix}
    B& C \\
    0& D
    \end{pmatrix}, \quad\text{где $B\in M_k$}
\end{gather}
\newline
\newline
\textbf{101. Собственный вектор линейного оператора.}\\
Пусть $\phi\colon V \rightarrow V$ --- линейный оператор.

Ненулевой вектор $v\in V$ называется \textit{собственным} для $V$, если $\phi(v) = \lambda v$ для некоторго $\lambda \in F$.
\newline
\newline
\textbf{102. Собственное значение линейного оператора.}\\
Элемент $\lambda \in F$ называется собственным значением линейного оператора $\phi$ векторно пространства $V$, если существует такой ненулевой вектор $v \in V$, что $\phi(v) =\lambda v$. 
\newline
\newline
\textbf{103. Спектр линейного оператора.} \\
   Ненулевой вектор $v \in V$ называется собственным для $V$, если $\varphi(v) = \lambda v$ для некоторго $\lambda \in F$. При этом число $\lambda$ называется собственным значением линейного оператора $\varphi$, отвечающим собственному вектору $v$. Множество собственных значений называется спектром оператора.
\newline
\newline
\textbf{104. Диагонализуемый линейный оператор.}\\
Линейный оператор $\phi\colon V \rightarrow V$ называется диагонализуемым, если существует базис $\mathbb{e}$ в $V$ такой, что $A(\phi, \mathbb{e})$ диагональна. 
\newline
\newline
\textbf{105. Критерий диагонализуемости линейного оператора в терминах собственных векторов.}\\
Линейный оператор $\phi\colon V \rightarrow V$ диагонализуем тогда и только тогда, когда в $V$ существует базис из собственных векторов.
\newline
\newline
\textbf{106. Собственное подпространство линейного оператора.}\\
Пусть $\phi\colon V \rightarrow V$ --- линейный оператор.

Множество $V_{\lambda}(\phi) = \{v\in V\ |\ \phi(v) = \lambda v\}$ называется собственным подпространством линейного оператора, отвечающим собственному значению $\lambda$.
\newline
\newline
\textbf{107. Характеристический многочлен линейного оператора.}\\
Пусть $\phi\colon V \rightarrow V$ --- линейный оператор.

Многочлен $\chi_{\phi}(t) = (-1)^ndet(\phi - t ~id)$ называется характеристическим для линейного оператора $\phi$.
\newline
\newline
\textbf{108. Связь спектра линейного оператора с его характеристическим многочленом.}
--
\newline
\newline
\textbf{109. Алгебраическая кратность собственного значения линейного оператора.}\\
Алгебраической кратностью собственного значения $\lambda$ линейного оператора $\phi \colon V \to V$ называется число $k$, которое равно кратности $\lambda$ как корня характеристического многочлена~$\phi$.
\newline
\newline
\textbf{110. Геометрическая кратность собственного значения линейного оператора.}\\
Пусть $\phi\colon V \rightarrow V$ --- линейный оператор, $\lambda$ --- его собственное значение и $V_\lambda(\phi)$ --- соответствующее собственное подпространство.

Геометрической кратностью собственного значения $\lambda$ называется число $dim V_\lambda(\phi)$.
\newline
\newline
\textbf{111. Связь между алгебраической и геометрической кратностями собственного значения линейного оператора.}\\
Геометрическая кратность не больше алгебраической кратности.
\newline
\newline
\textbf{112. Критерий диагонализуемости линейного оператора в терминах его характеристического многочлена и кратно-
стей его собственных значений.}\\
Пусть $\phi\colon V \rightarrow V$ --- линейный оператор.

Линейный оператор $\phi$ диагонализируем тогда и только тогда, когда 
\begin{enumerate}
\item $\chi_\phi(t)$ разлагается на линейные множители;
\item Если $\chi_\phi(t) = (t - \lambda_1)^{k_1}\dots(t - \lambda_s)^{k_s}$, где $\lambda_i \neq \lambda_j$ при $i \neq j$, то $\dim V_{\lambda_i}(\phi) = k_i \ \forall i$ (то есть для любого собственного значения $\phi$ равны геометрическая и алгебраическая кратности).
\end{enumerate}
\textbf{}
--
\newline
\newline
\textbf{}
--
\newline
\newline
\textbf{}
--
\newline
\newline

\end{document}







