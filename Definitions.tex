\documentclass{article}     
\usepackage[utf8]{inputenc} 
\usepackage{amsfonts}
\usepackage[left=4cm,right=4cm,
    top=3cm,bottom=4cm,bindingoffset=0cm]{geometry}
\sloppy
\usepackage[T2A]{fontenc}
\usepackage{amsmath, amssymb}
\title{Определения и формулировки теорем к коллоквиуму по линейной алгебре}  
\author{(часть определений - копипаста определений прошлого года!!1!)}     
\date{20 мая 2017}   

\usepackage{graphicx}
\graphicspath{{pictures/}}
\DeclareGraphicsExtensions{.pdf,.png,.jpg}

\newcommand{\ip}[2]{(#1, #2)}
                             
\begin{document}             
\maketitle
\noindent \textbf{1. Сумма двух подпространств векторного пространства.}\\

Пусть $V$ --- конечномерное векторное пространство, а $U$ и $W$ --- его подпространства.

Сумма подпространств $U$ и $W$ --- это множество
\[
U+W = \{u + w\ |\ u \in U, w \in W\},
\]
которое является подпространством векторного пространства $V$.
\newline
\newline
\textbf{2. Теорема о связи размерности суммы двух подпространств с размерностью их пересечения.}\\
Пусть $V$ --- конечномерное векторное пространство, а $U$ и $W$ --- его подпространства.

$\dim \left(U \cap W\right) = \dim U + \dim W - \dim \left(U+W\right)$
\\
\textbf{Прямая сумма двух подпространств векторного пространства.}\\
Пусть $V$ --- конечномерное векторное пространство, а $U$ и $W$ --- подпространства.

Если $U \cap W = \{0\}$, то $U + W$ называется прямой суммой.
\newline
\newline
\textbf{3. Сумма нескольких подпространств векторного пространства.}\\
Пусть $U_1, \ldots, U_k$ --- подпространства векторного пространства $V$. Суммой нескольких пространств называется $U_1 + \ldots + U_k = \{u_1 + \ldots + u_k \; | \; u_i \in U_i \}$.
\newline
\newline
\textbf{4. Линейная независимость нескольких подпространств векторного пространства.}
--
\newline
\newline
\textbf{5. Эквивалентные условия, определяющие линейно независимый набор подпространств векторного пространства.}
--
\newline
\newline
\textbf{6. Разложение векторного пространства в прямую сумму подпространств.}
--
\newline
\newline
\textbf{7. При каких условиях на подпространства U1,U2 векторного пространства V имеет место разложение $V =U_1\oplus U_2$?}
--
\newline
\newline
\textbf{8. Описание всех базисов n-мерного векторного пространства в терминах одного базиса и матриц координат.}
--
\newline
\newline
\textbf{9. Матрица перехода от одного базиса векторного пространства к другому.}\\
Пусть $V$ --- векторное пространство, $\dim V = n$, $e = (e_1, \ldots, e_n)$ и $e' =(e_1', \ldots, e_n')$ --- базисы в $V$.

Матрицей перехода от базисе $e$ к базису $e'$ называется матрица, по столбцам которой стоят координаты базиса $e'$ в базисе $e$.
\begin{gather*}
e'_j = \sum_{i = 1}^{n} c_{ij}e_i, \quad c_{ij} \in F \\
(e'_1, \ldots, e'_n) = (e_1, \ldots, e_n) \cdot C, \quad C = (c_{ij}) \text{--- матрица перехода}
\end{gather*}
\newline
\newline
\textbf{10. Формула преобразования координат вектора при замене базиса векторного пространства.}\\
Пусть $V$ --- векторное пространство. Формула преобразования координат вектора $v \in V$ при переходе от базиса $e$ к $e'$:
\begin{gather*}
\begin{pmatrix}
x_1 \\
\vdots \\
x_n
\end{pmatrix}
= C 
\begin{pmatrix}
x'_1 \\
\vdots \\
x'_n
\end{pmatrix}
\qquad \text{или} \qquad
x_i = \sum_{j = 1}^{n}c_{ij}x'_j,
\end{gather*}
где $(x_1, \ldots, x_n)$ --- координаты вектора $v$ в базисе $e $, $(x_1', \ldots, x_n')$ --- координаты вектора $v$  в базисе $e '$ и $C$ --- матрица перехода от базиса $e $ к базису $e '$.
\newline
\newline
\textbf{11. Линейное отображение векторных пространств, его простейшие свойства.}\\
Пусть $V$ и $W$ --- два векторных пространства над полем $F$.

Отображение $f : V \rightarrow W$ называется линейным, если:
\begin{enumerate}
\item $f(u_1 + u_2) = f(u_1) + f(u_2), \quad \forall u_1, u_2 \in V$;
\item $f(\alpha u) = \alpha f(u), \quad \forall u \in V,\ \forall \alpha \in F$.
\end{enumerate}
Его простейшие свойства:
\begin{enumerate}
     \item $f(\bar{0}_V)=\bar{0}_w$
     \item $f(-v) = -f(v) ~~\forall v \in V$
\end{enumerate}
\textbf{12. Изоморфизм векторных пространств. Изоморфные векторные пространства.}\\
Пусть $V$ и $W$ --- два векторных пространства над полем $F$.

Отображение $\phi: V \rightarrow W$ называется изоморфизмом, если $\phi$ линейно и биективно. Обозначение: $\phi : V \simeq  W$.

Два векторных пространства $V$ и $W$ называются изоморфными, если существует изоморфизм $\phi: V \simeq W$ (и тогда существует изоморфизм $V \backsimeq W$). Обозначение: $V \simeq W$ или $V \cong W$.
\newline
\newline
\textbf{13. Какими свойствами обладает отношение изоморфности на множестве всех векторных пространств?}\\
Изоморфность --- это отношение эквивалентности на множестве векторных пространств над фиксированным полем F. (т.е. оно рефлексивно, симметрично и транзитивно)
\newline
\newline
\textbf{14. Критерий изоморфности двух конечномерных векторных пространств.}\\
Два конечномерных векторных пространства $V$ и $W$ изоморфны тогда и только тогда, когда $dim V = dim W$.
\newline
\newline
\textbf{15. Матрица линейного отображения.}\\
Пусть $V$ и $W$ --- векторные пространства, $\mathbb{e} = (e_1, \ldots, e_n)$ --- базис $V$, $\mathbb{f} = (f_1, \ldots, f_m)$ --- базис $W$, $\phi: V \rightarrow W$ --- линейное отображение.

Матрицей линейного отображения $\phi$ в базисах $e$ и $\mathbb{f}$ (или по отношению к базисам $\mathbb{e}$ и~$\mathbb{f}$) называется такая матрица, у которой в $j$-ом столбце выписаны координаты вектора $\phi(e_j)$ в базисе $\mathbb{f}$.
\[
\phi(e_j) = a_{1j}f_1 + \ldots + a_{mj}f_m = \sum_{i = 1}^{m}a_{ij}f_i, \quad A = (a_{ij}) \in Mat_{m\times n} \text{ --- матрица $\phi$}
\]
\newline
\newline
\textbf{16. Связь между координатами вектора и его образа при линейном отображении.}\\
Пусть $e = (e_1, e_2...e_n)$ - базис в $V$, $f = (f_1, f_2...f_m)$ - базис в $W$. 
$$v = x_1e_1+x_2e_2+..+x_ne_n$$
$$\varphi \in Hom(V,W),~~~A=A(\varphi, e, f)$$
$$\varphi(v)=y_1f_1+y_2f_2+..+y_mf_m$$
Тогда:
$$\begin{pmatrix} y_1\\y_2\\\vdots\\y_m\end{pmatrix}=A\begin{pmatrix}x_1\\x_2\\\vdots\\x_n\end{pmatrix}
$$
\newline
\newline
\textbf{17. Формула изменения матрицы линейного отображения при замене базисов.}
Пусть $e = (e_1, e_2...e_n)$ - базис в $V$, $f = (f_1, f_2...f_m)$ - базис в $W$. А $e' = (e_1', e_2'...e_n')$, $f' = (f_1', f_2'...f_m')$ - другие базисы в $V$ и $W$ соответственно.
$$A=A(\varphi,e,f);~~A'=A(\varphi, e', f')$$
Пусть $e'=eC$, $f'=fD$. Тогда:
$$A'=D^{-1}AC$$
\newline
\newline
\textbf{18. Сумма двух линейных отображений и её матрица. Произведение линейного отображения на скаляр и его
матрица.}\\
Пусть $\phi, \psi \in Hom(V, W)$.

Отображение $\phi + \psi \in Hom(V, W)$ --- это $(\phi + \psi)(v):= \phi(v) + \psi(v)$ -- сумма отображений.

Пусть $\mathbb{e} = (e_1, \ldots, e_n)$ --- базис $V$, $\mathbb{f} = (f_1, \ldots, f_m)$ --- базис $W$, $\phi, \psi \in Hom(V, W)$. При этом $A_{\phi}$ --- матрица линейного отображения $\phi$, $A_{\psi}$ -Пусть $\phi, \psi \in Hom(V, W)$.

Отображение $\alpha \in F, \alpha\phi \in Hom(V, W)$ --- это $(\alpha\phi)(v) := \alpha(\phi(v))$ -- произведение линейного отображения на скаляр.

Пусть $\mathbb{e} = (e_1, \ldots, e_n)$ --- базис $V$, $\mathbb{f} = (f_1, \ldots, f_m)$ --- базис $W$, $\phi, \psi \in Hom(V, W)$. При этом $A_{\phi}$ --- матрица линейного отображения $\phi$, $A_{\psi}$ --- матрица для $\psi$,  $A_{\alpha\phi}$ --- для $\alpha\phi$.

Тогда $A_{\alpha\phi} = \alpha A_{\phi}$.-- матрица для $\psi$, $A_{\phi+\psi}$ --- для $\phi + \psi$.

Тогда $A_{\phi+\psi} = A_{\phi} + A_{\psi}$.
\newline
\newline
\textbf{19. Композиция двух линейных отображений и её матрица.}\\
Возьмем три векторных пространства --- $U, V$ и $W$ размерности $n, m$ и $k$ соответственно, и их базисы $\mathbb{e}, \mathbb{f}$ и $\mathbb{g}$. Также рассмотрим цепочку линейных отображений $U \xrightarrow{\psi} V \xrightarrow{\phi} W$. 

Отображение $\phi\circ\psi \in Hom(U, W)$ -- это $(\phi\circ\psi)(v) := \phi(\psi(v))$ -- композиция линейных отображений.

Пусть $A$ --- матрица $\phi$ в базисах $\mathbb{f}$ и $\mathbb{g}$, $B$ --- матрица $\psi$ в базисах $\mathbb{e}$ и $\mathbb{f}$, $C$ --- матрица $\phi\circ\psi$ в базисах $\mathbb{e}$ и $\mathbb{g}$.

Тогда $C = AB$.
\newline
\newline
\textbf{20. Ядро и образ линейного отображения.}\\
Пусть $V$ и $W$ --- векторные пространства, $\phi: V \rightarrow W$ --- линейное отображение.

\textit{Ядро $\phi$} --- это множество $Ker\phi := \{v \in V \mid \phi(v) = 0 \}$.

\textit{Образ $\phi$} --- это множество $Im \phi := \{w \in W \mid \exists v \in V : \phi(v) = w \}$.
\newline
\newline
\textbf{21. Критерий инъективности линейного отображения в терминах его ядра. Критерий изоморфности линейного
отображения в терминах его ядра и образа.}\\
Пусть $\phi\colon V \rightarrow W$ --- линейное отображение.

Отображение $\phi$ инъективно тогда и только тогда, когда $Ker \phi = \{0\}$.
Тогда $\dim Im \phi = rk A$.
\newline
\newline
\textbf{22. Связь между рангом матрицы линейного отображения и размерностью его образа.}\\
Пусть $V, W$ --- векторные пространства, $\mathbb{e} = (e_1, \ldots, e_n)$ --- базис $V$, $\mathbb{f} = (f_1, \ldots, f_m)$ --- базис $W$, $A$ --- матрица $\phi$ по отношению к $\mathbb{e},\ \mathbb{f}$.
\newline
\newline
\textbf{23. Оценки на ранг произведения двух матриц.}\\
Пусть $A \in Mat_{k\times m},\ B\in Mat_{m\times n}$. Тогда $rk AB \leqslant min\{rk A, rk B\}$.
\newline
\newline
\textbf{24. Каким свойством обладает набор векторов, дополняющих базис ядра линейного отображения до базиса всего пространства?}\\
Пусть  $\mathbb{e} = (e_1, \ldots, e_n)$ --- базис $V$, такой что $(e_1, \ldots, e_k)$ --- базис ядра $Ker\varphi$. Тогда $\varphi(e_{k+1}),\ldots\varphi(e_n)$ --- базис $Im\varphi$. 
Замечание: базис в $V$ с указанными условиями всегда существует.
\newline
\newline
\textbf{25. Теорема о связи размерностей ядра и образа линейного отображения.}\\
Пусть $\phi\colon V \rightarrow W$ --- линейное отображение.

Тогда $dim Im \phi = dim V - dim Ker \phi$.
\newline
\newline
\textbf{26. К какому простейшему виду можно привести матрицу линейного отображения путём замены базисов?}
--
\newline
\newline
\textbf{27. Линейная функция на векторном пространстве.}\\
Линейной функцией (формой, функционалом) на векторном пространстве $V$ называется всякое линейное отображение $\sigma \colon V \rightarrow F$.
\newline
\newline
\textbf{28. Сопряжённое (двойственное) векторное пространство и его размерность.}
--
\newline
\newline
\textbf{29. Базис сопряжённого пространства, двойственный к данному базису исходного векторного пространства.}\\
Пусть $e = (e_1, \ldots, e_n)$ --- базис $V$. Рассмотрим линейные формы $\epsilon_1, \ldots, \epsilon_n$ такие, что $\epsilon_i(e_j) =~\delta_{ij}$, где $\delta_{ij} =
\begin{cases}
1, & i = j \\
0, & i \neq j
\end{cases}
$ --- символ Кронекера. \\То есть $\epsilon_i = (\delta_{i1}, \ldots, \delta_{ii}, \ldots, \delta_{in}) = (0, \ldots, 1, \ldots, 0)$.

Тогда $(\epsilon_1, \ldots, \epsilon_n)$ --- базис в $V^*$, называющийся двойственным (сопряжённым) к базису $e$.
\newline
\newline
\textbf{30. Билинейная форма на векторном пространстве.}\\
Билинейной функцией (формой) на векторном пространстве $V$ называется всякое билинейное отображение $\beta \colon V \times V \rightarrow F$. То есть это отображение, линейное по каждому аргументу:
\begin{enumerate}
\item $\beta(x_1 + x_2, y) = \beta(x_1, y) + \beta(x_2, y)$; 
\item $\beta(\lambda x, y) = \lambda\beta(x, y)$;
\item $\beta(x, y_1 + y_2) = \beta(x, y_1) + \beta(x, y_2)$;
\item $\beta(x, \lambda y) = \lambda\beta(x, y)$.
\end{enumerate}
\textbf{31. Матрица билинейной формы.}\\
Пусть $V$ --- векторное пространство, $\dim V < \infty$, $\beta \colon V \times V \rightarrow F$ --- билинейная функция.

Матрицей билинейной функции в базисе $e = (e_1, \ldots, e_n)$ называется матрица $B = (b_{ij})$, где $b_{ij}= \beta(e_i, e_j)$.
\newline
\newline
\textbf{32. Формула для вычисления значений билинейной формы в координатах.}
Пусть $(e_1, \ldots, e_n)$ --- базис $V$, $\beta \colon V \times V \to F$ --- билинейная функция, $B$ --- ее матрица в базисе $e $. Тогда для некоторых векторов $x = x_1e_1 + \ldots + x_ne_n \in V$ и $y = y_1e_1 + \ldots + y_ne_n \in V$ верно, что:
$$
\beta(x, y) = (x_1, \ldots, x_n)B \begin{pmatrix}y_1\\ \vdots \\ y_n \end{pmatrix}
$$
\textbf{33. Формула изменения матрицы билинейной формы при переходе к другому базису.}
Пусть $e = (e_1, \ldots, e_n)$ и $e' = (e'_1, \ldots, e'_n)$ --- два базиса $V$, $\beta$ --- билинейная функция на $V$. Пусть также $e' = e C$, где $C$ --- матрица перехода, также $B(\beta, e) = B$ и $B(\beta, e') = B'$.

Тогда $B' = C^TBC$.
\newline
\newline
\textbf{34. Ранг билинейной формы.}
Пусть $B(\beta, e)$ -- матрица билинейной функции $\beta$ в базисе $e$.

Число $rk B$ называется рангом билинейной функции $\beta$. Обозначение: $rk \beta$.
\newline
\newline
\textbf{35. Симметричная билинейная форма.}
Билинейная функция $\beta$ называется симметричной, если $\beta(x, y) =\beta(y, x)$ для любых $x, y \in V$.
\newline
\newline
\textbf{36. Квадратичная форма.}
Пусть $\beta \colon V\times V \rightarrow F$ --- билинейная функция. Тогда функция $Q_\beta \colon V \rightarrow F$, заданная формулой $Q_\beta(x) = \beta(x, x)$, называется квадратичной формой (функцией), ассоциированной с билинейной функцией $\beta$.
\newline
\newline
\textbf{37. Соответствие между симметричными билинейными формами и квадратичными формами.}
Пусть $\beta \colon V\times V \rightarrow F$ --- симметричная билинейная функция, где $F$ --- поле, в котором $0 \neq 2$ (то есть можно делить на два).

Отображение $\beta(x, y) \mapsto Q_\beta(x) = \beta(x, x)$ является биекцией между симметричными билинейными функциями на $V$ и квадратичными функциями на $V$.
\newline
\newline
\textbf{38. Симметризация билинейной формы.}\\
--
\newline
\newline
\textbf{39. Поляризация квадратичной формы.}\\
Симметричная билинейная функция $\beta(x, y) = \frac{1}{2}\left( Q(x + y) - Q(x) - Q(y)  \right)$ называется поляризацией квадратичной формы $Q$.
\newline
\newline
\textbf{40. Матрица квадратичной формы.}\\
Пусть $V$ --- векторное пространство, $\dim V < \infty$.

Матрицей квадратичной формы $Q \colon V \to F$   в базисе $e$ называется матрица  соответствующей ей симметричной билинейной функции $\beta \colon V \times V \rightarrow F$ в том же базисе. 
\newline
\newline
\textbf{41. Канонический вид квадратичной формы.}\\
Квадратичная форма $Q$ имеет в базисе $e = (e_1, \ldots, e_n)$ канонический вид, если для любого вектора $x = x_1e_1 + \ldots + x_ne_n$ верно, что $Q(x) = a_1x_1^2 +\ldots + a_nx_n^2$, где $a_i \in F$. Иными словами, она имеет диагональную матрицу.
\newline
\newline
\textbf{42. Нормальный вид квадратичной формы над R.}\\
Квадратичная форма $Q$ имеет нормальный вид в базисе $e = (e_1,..,e_n)$, если для любого вектора $x = x_1e_1 + \ldots + x_ne_n$ верно, что $Q(x) = a_1x_1^2 +\ldots + a_nx_n^2$, причем $a_i \in \{-1, 0, 1\}$.  
\newline
\newline
\textbf{43. Индексы инерции квадратичной формы над R.}\\
Пусть $Q$ --- квадратичная форма над $\mathbb{R}$, которая в базисе $e =  (e_1,..,e_n)$ имеет нормальный вид: 
$$
Q(x_1, \ldots, x_n) = x_1^2 + \ldots + x_s^2 - x_{s + 1}^2 - \ldots - x_{s + t}^2,
$$
где $s$ --- это количество положительных слагаемых, а $t$ --- отрицательных. Тогда:

\begin{enumerate}
\item $i_+ := s$ --- положительный индекс инерции;
\item $i_- := t$ --- отрицательный индекс инерции;
\item $i_0 := n - s - t$ --- нулевой индекс инерции.
\end{enumerate}
\textbf{44. Закон инерции для квадратичной формы над R.}\\
Индексы инерции не зависят от выбора базиса, в котором квадратичная форма $Q$ имеет нормальный вид.
\newline
\newline
\textbf{45. Положительно/неотрицательно определённая квадратичная форма.}
Квадратичная форма $Q$ называется:
\begin{enumerate}
\item положительно определенной, если $Q(x) > 0\ \forall x \neq 0$. Обозначение: $Q > 0$;
\item неотрицательно определенной, если $Q(x) \geqslant 0\ \forall x$. Обозначение: $Q \geqslant 0$.
\end{enumerate}
\textbf{46. Отрицательно/неположительно определённая квадратичная форма.}\\
Квадратичная форма $Q$ называется:
\begin{enumerate}
\item отрицательно определенной, если $Q(x) < 0\  \forall x \neq 0$. Обозначение: $Q < 0$;
\item неположительно определенной, если $Q(x) \leqslant 0\ \forall x$. Обозначение: $Q \leqslant 0$.
\end{enumerate}
\textbf{47. Неопределённая квадратичная форма.}\\
Квадратичная форма $Q$ называется неопределенной, если существуют такие векторы $x, y$, что $Q(x) > 0$ и $Q(y) < 0$.
\newline
\newline
\textbf{48. Следствие метода Якоби о нахождении индексов инерции квадратичной формы.}
--
\newline
\newline
\textbf{49. Критерий Сильвестра положительной определённости квадратичной формы.}\\
Квадратичная форма $Q$ является положительно определенной тогда и только тогда, когда $\delta_i > 0$  для всех $i$.

Здесь $\delta_i$ --- угловой минор $i\times i$-подматрицы матрицы соответствующей симметричной билинейной функции $\beta$ в некотором базисе; $\delta_0 = 1$.
\newline
\newline
\textbf{50. Критерий отрицательной определённости квадратичной формы.}\\
Квадратичная форма $Q$ является отрицательно определенной тогда и только тогда,\\ когда:
$$
\begin{cases}
	\delta_i < 0, & 2 \nmid i \\
	\delta_i > 0, & 2 \mid i
\end{cases}.
$$
Здесь $\delta_i$ --- угловой минор $i\times i$-подматрицы матрицы соответствующей симметричной билинейной функции $\beta$ в некотором базисе; $\delta_0 = 1$.
\newline
\newline
\textbf{51. Евклидово пространство.}\\
Евклидово пространство --- это векторное пространство $\mathbb{E}$ над полем $\mathbb{R}$, на котором задана положительно определённая симметрическая билинейная функция $(\cdot, \cdot)$, которую мы будем называть скалярным произведением.
\newline
\newline
\textbf{52. Длина вектора в евклидовом пространстве.}
Пусть $\mathbb{E}$ --- евклидово пространство, $x\in \mathbb{E}$. Тогда длиной вектора называют величину $|x| = \sqrt{(x,x)}$.
\newline
\newline
\textbf{53. Неравенство Коши-Буняковского.}
Пусть $\mathbb{E}$ --- евклидово пространство, $x, y \in \mathbb{E}$. Тогда $|(x,y)| \leqslant |x||y|$, причём знак равенства возможен тогда и только тогда, когда $x$ и $y$ пропорциональны.
\newline
\newline
\textbf{55. Матрица Грама системы векторов евклидова пространства.}
Пусть $\mathbb{E}$ --- евклидово пространство.
Матрица Грама системы $v_1, \ldots, v_k \in \mathbb{E}$ это
	$$G(v_1,\ldots, v_k) := (g_{ij}),\quad g_{ij} = (v_i,v_j).$$
\newline
\newline
\textbf{56. Свойства определителя матрицы Грама.}
Пусть $\mathbb{E}$ --- евклидово пространство, $G(v_1,\ldots, v_k)$ --- матрица Грама. Тогда:
\begin{enumerate}
	\item $\det G(v_1, \ldots, v_k) \geqslant 0$;
	\item $\det G(v_1, \ldots, v_k) = 0$ тогда и только тогда, когда $v_1, \ldots, v_k$ линейно зависимы.
\end{enumerate}
\textbf{57. Ортогональное дополнение подмножества евклидова пространства.}
Пусть $S$ --- произвольное подпространство евклидова пространства $\mathbb{E}$. Ортогональным дополнением к $S$ называется множество $S^{\perp} = \{x\in \mathbb{E}\; |\; (x,y) = 0\;\forall y \in S\}$.
\newline
\newline
\textbf{58. Чему равна размерность ортогонального дополнения к подпространству?}
--
\newline
\newline
\textbf{59. Каким свойством обладают подпространство евклидова пространства и его ортогональное дополнение?}
--
\newline
\newline
\textbf{60. Ортогональная проекция вектора на подпространство.}
Пусть $S$ --- подпространство евклидова пространства $\mathbb{E}$. Тогда любой вектор $x \in E$ единственным образом разбивается на сумму $x = y + z$, где $y \in S$ и $z \in S^\perp$. 

Вектор $y$ называется ортогональной проекцией вектора $x$ на подпространство $S$. Обозначение: $pr_S x$. 
\newline
\newline
\textbf{61. Ортогональная составляющая вектора относительно подпространства.}
Пусть $S$ --- подпространство евклидова пространства $\mathbb{E}$. Тогда любой вектор $x \in E$ единственным образом разбивается на сумму $x = y + z$, где $y \in S$ и $z \in S^\perp$. 

Вектор $z$ называется ортогональной составляющей вектора $x$ относительно (вдоль) подпространства $S$. Обозначение: $ort_S x$.
\newline
\newline
\textbf{62. Формула для ортогональной проекции вектора на подпространство в Rn, заданное своим базисом.}
--
\newline
\newline
\textbf{63. !!!Ортогональная система векторов. Ортогональный базис.}
Базис $(e_1, \ldots, e_n)$ в евклидовом пространстве $\mathbb{E}$ называется ортогональным, если \\$(e_i, e_j)= 0\; \forall i\neq j$. Это равносильно тому, что $G(e_1, \ldots, e_n)$ диагональна.
\newline
\newline
\textbf{64. !!!Ортонормированная система векторов. Ортонормированный базис.}
Базис $(e_1, \ldots, e_n)$ в евклидовом пространстве $\mathbb{E}$ называется ортонормированным, если он является ортогональным базисом и дополнительно $(e_i, e_i) = 1\; \forall i$. Это равносильно тому, что $G(e_1, \ldots, e_n) = E$.
\newline
\newline
\textbf{65. Описание всех ортонормированных базисов евклидова пространства в терминах одного такого базиса и матриц перехода.}
--
\newline
\newline
\textbf{66. Ортогональная матрица.}
Пусть $e = (e_1, \ldots, e_n)$ и $e' = (e_1', \ldots, e_n')$ --- два ортонормированных базиса в евклидовом пространстве $\mathbb{E}$, причем $(e_1', \ldots, e_n') = (e_1, \ldots, e_n)C$. 

Тогда матрица $C$ называется ортогональной.
\newline
\newline
\textbf{67. Формула для ортогональной проекции вектора на подпространство в терминах его ортогонального базиса.}
Пусть $S$ --- подпространство евклидова пространства $\mathbb{E}$, $(e_1, \ldots, e_k)$ --- его ортогональный базис, $x \in \mathbb{E}$.
Тогда $pr_S x = \sum\limits_{i = 1}^{k}\cfrac{(x,e_i)}{(e_i, e_i)}e_i$. В частности, если базис ортонормированный, $pr_S x = \sum\limits_{i = 1}^{k}(x,e_i)e_i$
\newline
\newline
\textbf{68. Теорема Пифагора в евклидовом пространстве.}
Если векторы евклидова пространства $x$ и $y$ перпендикулярны, то $|x+y|^2 = |x|^2 + |y|^2$.
\newline
\newline
\textbf{69. Расстояние между векторами евклидова пространства.}
Расстоянием между векторами евклидова пространства $x$ и $y$ называется число \\$\rho(x,y) := |x-y|$.
\newline
\newline
\textbf{70. Неравенство треугольника в евклидовом пространстве.}
--
\newline
\newline
\textbf{71. Теорема о расстоянии между вектором и подпространством в терминах ортогональной составляющей.}
Пусть $U$ --- подпространство евклидова пространства $\mathbb{E}$.

Модуль ортогональной составляющей вектора $x \in E$ относительно подпространства $U$ равен расстоянию от вектора $x$ до подпространства $U$.
$$
\rho (x,U) = |ort_U x|.
$$
\newline
\newline
\textbf{72. Псевдорешение несовместной системы линейных уравнений.}
--
\newline
\newline
\textbf{73. Формула для расстояния от вектора до подпространства в терминах матриц Грама.}
Пусть $U$ --- подпространство евклидова пространства $\mathbb{E}$, $x \in \mathbb{E}$, $(e_1, \ldots, e_k)$ --- базис $U$. Тогда 
$(\rho(x,U))^2 = \cfrac{\det G(e_1, \ldots, e_k, x)}{\det G(e_1, \ldots, e_k)}$.
\newline
\newline
\textbf{74. n-мерный параллелепипед и его объём.}
$N$-мерным параллелепипедом, натянутым на векторы $a_1, \ldots, a_n$ евклидова пространства $\mathbb{E}$ называется множество 
$$
P(a_1, \ldots, a_n) := \left\{ x = \sum_{i = 1}^n x_ia_i \mid 0 \leqslant x_i \leqslant 1 \right\}.
$$  
Объем $n$-мерного параллелепипеда $P(a_1, \ldots, a_{n})$ --- это число $vol P(a_1, \ldots, a_{n})$, определяемое рекурсивно следующим образом:
\begin{align*}
n = 1 \quad& vol P(a_1) = |a_1| \\
n > 1 \quad& vol P(a_1, \ldots, a_n) = vol P(a_1, \ldots, a_{n-1})\cdot |h|
\end{align*}
Где $h = ort_{\langle a_1, \ldots, a_{n-1}\rangle}a_n$ --- высота $P(a_1, \ldots, a_n)$.
Пусть $\mathbb{E}$ --- векторное пространство, $(e_1,..,e_n)$ --- его ортогональный базис и $(a_1,..,a_n) = (e_1,..,e_n)A$ для некоторой матрицы $A \in M_n(\mathbb{R})$. Тогда $vol P(a_1,..,a_n) = |det A|$.
\newline
\newline
\textbf{75. В каком случае два базиса евклидова пространства называются одинаково ориентированными?}
--
\newline
\newline
\textbf{76. Смешанное произведение векторов трёхмерного евклидова пространства, формула для его вычисления в терминах координат в правом ортонормированном базисе.}
--
\newline
\newline
\textbf{77. Критерий компланарности трёх векторов трёхмерного евклидова пространства.}
--
\newline
\newline
\textbf{78. Векторное произведение в трёхмерном евклидовом пространстве.}
--
\newline
\newline
\textbf{79. Критерий коллинеарности двух векторов трёхмерного евклидова пространства.}
--
\newline
\newline
\textbf{80. Выражение смешанного произведения через векторное и скалярное в трёхмерном евклидовом пространстве.}
--
\newline
\newline
\textbf{81. Формула для двойного векторного произведения в трёхмерном евклидовом пространстве.
}
--
\newline
\newline
\textbf{82. Формула для вычисления векторного произведения в терминах координат в правом ортонормированном базисе.}
--
\newline
\newline
\textbf{83. Линейное многообразие. Характеризация линейных многообразий как сдвигов подпространств.}
--
\newline
\newline
\textbf{84. Критерий равенства двух линейных многообразий. Направляющее подпространство и размерность линейного
многообразия.}
--
\newline
\newline
\textbf{85. Теорема о плоскости, проходящей через k + 1 точку в Rn.}
--
\newline
\newline
\textbf{86. Три способа задания прямой в R2. Уравнение прямой в R2, проходящей через две различные точки.}
--
\newline
\newline
\textbf{87. Три способа задания плоскости в R3. Уравнение плоскости в R3, проходящей через три точки, не лежащие на
одной прямой.}
--
\newline
\newline
\textbf{88. Три способа задания прямой в R3. Уравнения прямой в R3, проходящей через две различные точки.}
--
\newline
\newline
\textbf{89. Случаи взаимного расположения двух прямых в R3.}
--
\newline
\newline
\textbf{90. Случаи взаимного расположения трёх попарно различных плоскостей в R3.}
--
\newline
\newline
\textbf{91. Формула для расстояния от точки до прямой в R3.}
--
\newline
\newline
\textbf{92. Формула для расстояния от точки до плоскости в R3.}
--
\newline
\newline
\textbf{93. Формула для расстояния между двумя скрещивающимися прямыми в R3.}
--
\newline
\newline
\textbf{94. Линейный оператор.}\\
Пусть $V$ --- конечномерное векторное пространство.

Линейным оператором (или линейным преобразованием) называется всякое линейное отображение $\phi \colon V \rightarrow V$, то есть из $V$ в себя.
\newline
\newline
\textbf{95. Матрица линейного оператора.}
Пусть $V$ --- векторное пространство, $\mathbb{e} = (e_1, \ldots, e_n)$ --- его базис и $\phi$ --- его линейный оператор.

Матрицей линейного оператора $\phi$ называется такая матрица, в $j$-ом столбце которой стоят координаты вектора $\phi(e_j)$ в базисе $e$.
$$
\left(\phi(e_1), \ldots, \phi(e_n)\right) = \left(e_1, \ldots, e_n\right)A, \quad A \text{ --- матрица $\phi$}
$$
\textbf{96. Формула преобразования координат вектора при действии линейного оператора.}
--
\newline
\newline
\textbf{97. Формула изменения матрицы линейного оператора при переходе к другому базису.}\\
Пусть $\phi$ --- линейный оператор векторного пространства $V$, $A$ --- матрица $\phi$ в базисе $\mathbb{e} = (e_1, \ldots, e_n)$. Пусть $\mathbb{e}' = (e_1', \ldots, e_n')$ --- другой базис, причём $(e_1', \ldots, e_n') = (e_1, \ldots, e_n)C$, где $C$ --- матрица перехода, и $A'$ --- матрица $\phi$ в базисе $\mathbb{e}'$.

Тогда $A' = C^{-1}AC$.
\newline
\newline
\textbf{98. Подобные матрицы.}\\
Две матрицы $A', A \in M_n(F)$ называются подобными, если существует такая матрица $C \in M_n(F), det C \neq 0$, что $A' = C^{-1}AC$.
\newline
\newline
\textbf{99. Подпространство, инвариантное относительно линейного оператора.}\\
Пусть $\phi\colon V \rightarrow V$ --- линейный оператор.

Подпространство $U \subseteq V$ называется инвариантным относительно $\phi$ (или $\phi$-инвариантным), если $\phi(U)\subseteq U$. То есть $\forall u\in U \colon \phi(u)\in U$. 
\newline
\newline
\textbf{100. Матрица линейного оператора в базисе, дополняющем базис инвариантного подпространства.}\\
Пусть $\phi\colon V \rightarrow V$ --- линейный оператор.

Пусть $U\subset V$ --- $\phi$-инвариантное подпространство. Также пусть $(e_1, \ldots, e_k)$ --- базис в $U$. Дополним его до базиса $V\colon$ $\mathbb{e} = (e_1, \ldots, e_n)$. Тогда
\begin{gather}
    \underbrace{A(\phi,\;\mathbb{e})}_{\text{Матрица с углом нулей}} = \begin{pmatrix}
    B& C \\
    0& D
    \end{pmatrix}, \quad\text{где $B\in M_k$}
\end{gather}
\newline
\newline
\textbf{101. Собственный вектор линейного оператора.}\\
Пусть $\phi\colon V \rightarrow V$ --- линейный оператор.

Ненулевой вектор $v\in V$ называется \textit{собственным} для $V$, если $\phi(v) = \lambda v$ для некоторго $\lambda \in F$.
\newline
\newline
\textbf{102. Собственное значение линейного оператора.}\\
Элемент $\lambda \in F$ называется собственным значением линейного оператора $\phi$ векторно пространства $V$, если существует такой ненулевой вектор $v \in V$, что $\phi(v) =\lambda v$. 
\newline
\newline
\textbf{103. Спектр линейного оператора.}
--
\newline
\newline
\textbf{104. Диагонализуемый линейный оператор.}\\
Линейный оператор $\phi\colon V \rightarrow V$ называется диагонализуемым, если существует базис $\mathbb{e}$ в $V$ такой, что $A(\phi, \mathbb{e})$ диагональна. 
\newline
\newline
\textbf{105. Критерий диагонализуемости линейного оператора в терминах собственных векторов.}\\
Линейный оператор $\phi\colon V \rightarrow V$ диагонализуем тогда и только тогда, когда в $V$ существует базис из собственных векторов.
\newline
\newline
\textbf{106. Собственное подпространство линейного оператора.}\\
Пусть $\phi\colon V \rightarrow V$ --- линейный оператор.

Множество $V_{\lambda}(\phi) = \{v\in V\ |\ \phi(v) = \lambda v\}$ называется собственным подпространством линейного оператора, отвечающим собственному значению $\lambda$.
\newline
\newline
\textbf{107. Характеристический многочлен линейного оператора.}\\
Пусть $\phi\colon V \rightarrow V$ --- линейный оператор.

Многочлен $\chi_{\phi}(t) = (-1)^ndet(\phi - t ~id)$ называется характеристическим для линейного оператора $\phi$.
\newline
\newline
\textbf{108. Связь спектра линейного оператора с его характеристическим многочленом.}
--
\newline
\newline
\textbf{109. Алгебраическая кратность собственного значения линейного оператора.}\\
Алгебраической кратностью собственного значения $\lambda$ линейного оператора $\phi \colon V \to V$ называется число $k$, которое равно кратности $\lambda$ как корня характеристического многочлена~$\phi$.
\newline
\newline
\textbf{110. Геометрическая кратность собственного значения линейного оператора.}\\
Пусть $\phi\colon V \rightarrow V$ --- линейный оператор, $\lambda$ --- его собственное значение и $V_\lambda(\phi)$ --- соответствующее собственное подпространство.

Геометрической кратностью собственного значения $\lambda$ называется число $dim V_\lambda(\phi)$.
\newline
\newline
\textbf{111. Связь между алгебраической и геометрической кратностями собственного значения линейного оператора.}\\
Геометрическая кратность не больше алгебраической кратности.
\newline
\newline
\textbf{112. Критерий диагонализуемости линейного оператора в терминах его характеристического многочлена и кратно-
стей его собственных значений.}\\
Пусть $\phi\colon V \rightarrow V$ --- линейный оператор.

Линейный оператор $\phi$ диагонализируем тогда и только тогда, когда 
\begin{enumerate}
\item $\chi_\phi(t)$ разлагается на линейные множители;
\item Если $\chi_\phi(t) = (t - \lambda_1)^{k_1}\dots(t - \lambda_s)^{k_s}$, где $\lambda_i \neq \lambda_j$ при $i \neq j$, то $\dim V_{\lambda_i}(\phi) = k_i \ \forall i$ (то есть для любого собственного значения $\phi$ равны геометрическая и алгебраическая кратности).
\end{enumerate}
\textbf{}
--
\newline
\newline
\textbf{}
--
\newline
\newline
\textbf{}
--
\newline
\newline

\end{document}
